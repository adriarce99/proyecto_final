\documentclass[a4paper,11pt]{article}
\usepackage[utf8]{inputenc}
\usepackage[spanish]{babel}

\begin{document}

\begin{titlepage}
\title{Ley de Ohm y teorema de Thevenin}
\author{Adrián Arce Sánchez}
\maketitle
\end{titlepage}

\begin{abstract}

En este informe se realiza el estudio del circuito de un entrenador mediante la ley de Ohm. Una vez estudiado se hará uso de este para probar el teorema de Thevenin.
\end{abstract}

\section{Fundamento Teórico}
Uno de los conceptos clabe a la hora de hablar de un circuito es la ley de Ohm. Esta, establece una relación de proporcionalidad entre la corriente que circula por un elemento circuital y la diferencia de potencial en los extremos del mismo. La constante de proporcionalidad que rige esta ley se denomina la esistencia. Se trata de una propiedad intrinseca de cada elemento circuital siendo la capacidad de oponerse al flujo de corriente que lo atraviesa. Esta propiedad depende de las dimensones del material con el que se ha construido dicho circuito.

\begin{displaymath}
V_{0}=RI
\end{displaymath}

Cuando por dos o más elementos circuitales fluye la misma corriente se dice que estos elementos están conectados en serie. En estos casos es posible crear un circuito equivalente en el cual todos los elementos que presentan resistencia a la corriente se representan mediante una unica resistencia cuyo valor queda definido como la suma de todas las resistencias de los elementos por separado.

\begin{displaymath}
R_{serie}=\sum_{i=1}^N R_{i}
\end{displaymath}

Además, podemos declarar a partir de las resistencias en serie, una asociación de resistencias. Capaces de repartir la tensión suminnistrada a cada uno de los ementos de la asociación. De este modo, la tensión "Vi" que el elemento i recibirá vendrá dado por la ecuación:

\begin{displaymath}
R_{serie}=\sum_{i=1}^N R_{i}
\end{displaymath}

\end{document}